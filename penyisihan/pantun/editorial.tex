% !TeX root = ../main_editorial.tex
\documentclass[../main_editorial.tex]{subfiles} % Inherits definitions from parent .tex file.

% Per-problem variable definitions
\newcommand{\problemName}{Berbalas Pantun}
\newcommand{\problemWriter}{Alham Fikri Aji}
\newcommand{\problemEditorialWriter}{Anthony}
\newcommand{\problemTags}{\textit{ad hoc}}

\begin{document}

\begin{center}
    \section*{\problemName}
    \addcontentsline{toc}{section}{\problemName} % for pdf indexing
    
    \begin{tabular}{rl}
    Penulis soal : & \problemWriter \\
    Penulis editorial : & \problemEditorialWriter \\
    Tema : & \problemTags
    \end{tabular}
\end{center}

\subsection*{Catatan/Komentar}
\addcontentsline{toc}{subsection}{Catatan/Komentar} % for pdf indexing

Soal ini merupakan soal penyisihan termudah karena tidak dibutuhkan pengetahuan khusus untuk menyelesaikannya.

Batasan kedua versi: $1 \le A[i], B[i] \le 100$

\subsection*{Versi Mudah}
\addcontentsline{toc}{subsection}{Versi Mudah} % for pdf indexing
Batasan: $\mathbf{N = 1}$

Untuk versi mudah dari soal ini, karena $N = 1$, maka jawabannya adalah $A[1] + B[1]$.

\subsection*{Versi Sulit}
\addcontentsline{toc}{subsection}{Versi Sulit} % for pdf indexing

Batasan: $\mathbf{1 \le N \le 100.000}$

Berikut adalah jumlahan yang diminta pada soal:
\begin{gather*}
(A[1] + B[1]) + (A[1] + B[2]) + (A[1] + B[3]) + \dots + (A[1] + B[N])\\
+(A[2] + B[1]) + (A[2] + B[2]) + (A[2] + B[3]) + \dots + (A[2] + B[N])\\
+ \dots \\
+ (A[N] + B[1]) + (A[N] + B[2]) + (A[N] + B[3]) + \dots + (A[N] + B[N])
\end{gather*}
Setiap elemen dari $A$ akan dipasangkan dengan setiap elemen dari $B$, sehingga jumlahan untuk sebuah $A[i]$ dapat dituliskan sebagai berikut:
\begin{align*}
(A[i] + B[1]) + (A[i] + B[2]) + \dots + (A[i] + B[N]) &= (A[i] \times N) + (B[1] + B[2] + \dots + B[N])\\
&= (A[i] \times N) + \sum_{j=1}^{N}B[j]
\end{align*}
Jawaban akhir diperoleh dengan menjumlahkan persamaan di atas untuk setiap $i$ dari 1 hingga $N$:
\begin{align*}
\sum_{i=1}^{N} \left((A[i] \times N) + \sum_{j=1}^{N}B[j]\right) &= N \times \left(\sum_{j=1}^{N}B[j]\right) + \sum_{i=1}^{N} (A[i] \times N)\\
&= N \times \left(\sum_{j=1}^{N}B[j]\right) + N \times \left(\sum_{i=1}^{N} A[i]\right)\\
&= N \times \left(\sum_{i=1}^{N}A[i] + \sum_{j=1}^{N}B[j]\right)
\end{align*}
Rumus jawaban akhir menjadi sangat sederhana, yakni: N dikalikan dengan jumlahan semua elemen-elemen A dan B.

\end{document}
