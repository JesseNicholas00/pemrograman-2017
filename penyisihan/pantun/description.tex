% !TeX root = ../main_problemset.tex
\documentclass[../main_problemset.tex]{subfiles} % Inherits definitions from parent .tex file.

% Per-problem variable definitions
\newcommand{\problemName}{A. Berbalas Pantun}
\newcommand{\problemTL}{1 s}
\newcommand{\problemML}{64 MB}

\begin{document}

\begin{center}
    \section*{\problemName}
    \addcontentsline{toc}{section}{\problemName} % for pdf indexing

    \begin{tabular}{rl}
    Batasan waktu : & \problemTL \\
    Batasan memori : & \problemML
    \end{tabular}
\end{center}

\subsection*{Deskripsi}
\addcontentsline{toc}{subsection}{Deskripsi} % for pdf indexing

Murid-murid kelas 6 SD Chanek terbagi atas kelas 6A dan kelas 6B, yang masing-masing terdiri atas $N$ murid. Murid ke-$i$ di kelas 6A memiliki pantun sepanjang $A[i]$ detik, dan murid ke-$i$ di kelas 6B memiliki pantun sepanjang $B[i]$ detik.

Pada acara perpisahan, setiap murid di kelas 6A akan berbalas pantun dengan setiap murid di kelas 6B. Untuk sepasang murid ke-$i$ di kelas 6A dan murid ke-$j$ di kelas 6B, total waktu yang dibutuhkan mereka untuk berbalas pantun adalah $A[i] + B[j]$.

Panggung perpisahan hanya dapat menampilkan sepasang murid untuk berbalas pantun dalam satu waktu. Tentukan total waktu yang dibutuhkan seluruh kemungkinan pasang murid kelas 6A dan 6B untuk berbalas pantun pada panggung.

\subsection*{Format Masukan}
\addcontentsline{toc}{subsection}{Format Masukan} % for pdf indexing

Masukan diberikan dalam format berikut ini:

\begin{lcverbatim}
N
A[1] A[2] .. A[N]
B[1] B[2] .. B[N]
\end{lcverbatim}

\subsection*{Format Keluaran}
\addcontentsline{toc}{subsection}{Format Keluaran} % for pdf indexing

Keluarkan sebuah baris berisi total waktu yang dibutuhkan, dalam detik.

\vspace*{.4cm}

\begin{minipage}[t]{0.5\textwidth}
\subsection*{Contoh Masukan 1}
\addcontentsline{toc}{subsection}{Contoh Masukan 1} % for pdf indexing

\begin{lcverbatim}
1
3
5
\end{lcverbatim}
\end{minipage}
\begin{minipage}[t]{0.5\textwidth}
\subsection*{Contoh Keluaran 1}
\addcontentsline{toc}{subsection}{Contoh Keluaran 1} % for pdf indexing

\begin{lcverbatim}
8
\end{lcverbatim}
\end{minipage}

\vspace*{.4cm}

\begin{minipage}[t]{0.5\textwidth}
\subsection*{Contoh Masukan 2}
\addcontentsline{toc}{subsection}{Contoh Masukan 2} % for pdf indexing

\begin{lcverbatim}
2
1 2
3 4
\end{lcverbatim}
\end{minipage}
\begin{minipage}[t]{0.5\textwidth}
\subsection*{Contoh Keluaran 2}
\addcontentsline{toc}{subsection}{Contoh Keluaran 2} % for pdf indexing

\begin{lcverbatim}
20
\end{lcverbatim}
\end{minipage}

\textit{Perhatikan bahwa contoh kedua tidak termasuk dalam contoh masukan dan contoh keluaran dari soal versi mudah.}

\subsection*{Penjelasan}
\addcontentsline{toc}{subsection}{Penjelasan} % for pdf indexing

Untuk contoh pertama, terdapat 1 kemungkinan pasangan murid yang akan tampil: satu-satunya murid kelas 6A dan satu-satunya murid kelas 6B. Total waktu yang dibutuhkan adalah 3 + 5 = 8 detik.

Untuk contoh kedua, terdapat 4 kemungkinan pasangan murid yang akan tampil:
\begin{itemize}
	\item murid 1 kelas 6A dan murid 1 kelas 6B; total waktu = 1 + 3 = 4 detik
	\item murid 1 kelas 6A dan murid 2 kelas 6B; total waktu = 1 + 4 = 5 detik
	\item murid 2 kelas 6A dan murid 1 kelas 6B; total waktu = 2 + 3 = 5 detik
	\item murid 2 kelas 6A dan murid 2 kelas 6B; total waktu = 2 + 4 = 6 detik
\end{itemize}
Total waktu yang dibutuhkan adalah 4 + 5 + 5 + 6 = 20 detik.

\subsection*{Batasan}
\addcontentsline{toc}{subsection}{Batasan} % for pdf indexing

\begin{minipage}[t]{0.47\textwidth}
    
Batasan yang berlaku untuk versi mudah dan versi sulit:

\begin{itemize}
	\item $1 \le A[i], B[i] \le 100$
\end{itemize}
\end{minipage}
\begin{minipage}[t]{0.06\textwidth}
    \hfill
\end{minipage}
\begin{minipage}[t]{0.47\textwidth}
Batasan khusus versi mudah:

\begin{itemize}
	\item $N = 1$
\end{itemize}

\vspace{.2cm}

Batasan khusus versi sulit:

\begin{itemize}
	\item $1 \le N \le 100.000$
\end{itemize}
\end{minipage}

\end{document}
