% !TeX root = ../main_problemset.tex
\documentclass[../main_problemset.tex]{subfiles} % Inherits definitions from parent .tex file.

% Per-problem variable definitions
\newcommand{\problemName}{F. Rubrik Petakata}
\newcommand{\problemTL}{3 s}
\newcommand{\problemML}{64 MB}

\begin{document}

\begin{center}
    \section*{\problemName}
    \addcontentsline{toc}{section}{\problemName} % for pdf indexing
    
    \begin{tabular}{rl}
    Batasan waktu : & \problemTL \\
    Batasan memori : & \problemML
    \end{tabular}
\end{center}

\subsection*{Deskripsi}
\addcontentsline{toc}{subsection}{Deskripsi} % for pdf indexing

Petakata adalah sebuah rubrik teka-teki pada koran langganan Pak Chanek. Sebuah petakata terdiri atas petak-petak berukuran $ 2 $ baris dan $ N $ kolom. Setiap petak berisi sebuah huruf dari $ K $ huruf kecil pertama pada alfabet.

Tujuan pembaca adalah untuk menemukan sebuah string $ S $ pada petakata tersebut. Sebuah string $ S $ dikatakan terdapat pada petakata, apabila terdapat serangkaian petak yang memenuhi seluruh syarat di bawah ini:

\begin{itemize}
	\item dimulai pada sebuah petak yang mana saja,
	\item setiap petak, selain petak pertama, berada tepat di sebelah atas, bawah, kanan, kanan-atas, atau kanan-bawah dari petak sebelumnya (perhatikan bahwa setiap petak \textbf{tidak boleh berada di sebelah kiri} petak sebelumnya),
	\item tidak ada petak yang dikunjungi lebih dari satu kali, dan
	\item huruf-huruf pada petak-petak yang dikunjungi tepat membentuk string $ S $ (yakni, petak pertama berisi huruf pertama dari $ S $, petak kedua berisi huruf kedua, dan seterusnya).
\end{itemize}

Merasa bosan mengerjakan petakata setiap minggu, Pak Chanek malah penasaran: diberikan $ S $, $ N $, dan $ K $, ada berapa petakata berbeda sedemikian sehingga pembaca dapat menemukan string $ S $ pada petakata? Dua buah petakata dikatakan berbeda apabila terdapat setidaknya sebuah petak pada posisi yang sama namun berisi huruf yang berbeda di antara kedua petakata tersebut.

Bantulah Pak Chanek menghitung banyaknya petakata yang dimaksud, modulo $ 1.000.000.007 $.

\subsection*{Format Masukan}
\addcontentsline{toc}{subsection}{Format Masukan} % for pdf indexing

Baris pertama berisi sebuah bilangan bulat $ T $ yang menyatakan banyaknya kasus uji. Baris-baris berikutnya berisi $ T $ kasus uji, yang masing-masing diberikan dalam format berikut ini:

\begin{lcverbatim}
S N K
\end{lcverbatim}

\subsection*{Format Keluaran}
\addcontentsline{toc}{subsection}{Format Keluaran} % for pdf indexing

Untuk setiap kasus uji, keluarkan sebuah baris berisi banyaknya petakata yang mungkin, modulo $ 1.000.000.007 $.

\vspace{.4cm}

\begin{minipage}[t]{0.5\textwidth}
\subsection*{Contoh Masukan}
\addcontentsline{toc}{subsection}{Contoh Masukan} % for pdf indexing

\begin{lcverbatim}
4
ab 1 2
aa 2 2
gemastik 3 21
gemastik 20 21
\end{lcverbatim}
\end{minipage}
\begin{minipage}[t]{0.5\textwidth}
\subsection*{Contoh Keluaran}
\addcontentsline{toc}{subsection}{Contoh Keluaran} % for pdf indexing

\begin{lcverbatim}
2
11
0
662038843
\end{lcverbatim}
\end{minipage}

\pagebreak
\subsection*{Penjelasan}
\addcontentsline{toc}{subsection}{Penjelasan} % for pdf indexing

Untuk contoh pertama, terdapat 2 petakata yang mungkin:

\begin{lcverbatim}
a  |  b
b  |  a
\end{lcverbatim}

Untuk contoh kedua, terdapat 11 petakata yang mungkin:

\begin{lcverbatim}
 aa  |  aa  |  aa  |  aa
 aa  |  ab  |  ba  |  bb
-----+------+------+-----
 ab  |  ab  |  ab  |  ba
 aa  |  ab  |  ba  |  aa
-----+------+------+-----
 ba  |  ba  |  bb
 ab  |  ba  |  aa
\end{lcverbatim}

Untuk contoh ketiga, string gemastik tidak mungkin terdapat pada petakata berukuran $ 2 \times 3 $.

\subsection*{Batasan}
\addcontentsline{toc}{subsection}{Batasan} % for pdf indexing

Batasan yang berlaku untuk versi mudah dan versi sulit:

\begin{itemize}
	\item $ 1 \le T \le 5 $
	\item $ 1 \le N \le 50 $
	\item $ 1 \le K \le 26 $
	\item Setiap huruf pada $ S $ termasuk dalam $ K $ huruf kecil pertama pada alfabet
\end{itemize}

Batasan khusus versi mudah:

\begin{itemize}
	\item $ S $ terdiri atas tepat 2 huruf
\end{itemize}

Batasan khusus versi sulit:

\begin{itemize}
	\item $ S $ terdiri atas 1 hingga 10 huruf, inklusif
\end{itemize}

\end{document}
