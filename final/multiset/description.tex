% !TeX root = ../main_problemset.tex
\documentclass[../main_problemset.tex]{subfiles} % Inherits definitions from parent .tex file.

% Per-problem variable definitions
\newcommand{\problemName}{E. Pak Grandi}
\newcommand{\problemTL}{3 s}
\newcommand{\problemML}{256 MB}

\begin{document}

\begin{center}
    \section*{\problemName}
    \addcontentsline{toc}{section}{\problemName} % for pdf indexing
    
    \begin{tabular}{rl}
    Batasan waktu : & \problemTL \\
    Batasan memori : & \problemML
    \end{tabular}
\end{center}

\subsection*{Deskripsi}
\addcontentsline{toc}{subsection}{Deskripsi} % for pdf indexing

Pak Grandi memiliki papan tulis yang besar sekali.

Pak Chanek menantang Pak Grandi untuk melakukan N aksi, yang masing-masing berbentuk:

\begin{lcverbatim}
P[i] L[i] R[i] K[i]
\end{lcverbatim}

Makna dari aksi di atas adalah sebagai berikut:

\begin{itemize}
	\item Jika P[i] = 1: Pak Grandi harus menuliskan bilangan x sebanyak tepat K[i] buah pada papan tulis, untuk seluruh bilangan bulat L[i] $ \le $ x $ \le $ R[i].
	\item Jika P[i] = 2: Misalkan cnt(x) menyatakan banyaknya bilangan x pada papan tulis. Pak Grandi harus menghapus bilangan x sebanyak tepat min(cnt(x), K[i]) buah dari papan tulis, untuk seluruh bilangan bulat L[i] $ \le $ x $ \le $ R[i].
\end{itemize}

Segera setelah melakukan aksi yang diminta, Pak Grandi harus menghitung G[i], yang merupakan bilangan bulat positif terkecil yang tidak tertulis pada papan tulis.

Bantulah Pak Grandi untuk melakukan tantangan yang diminta Pak Chanek.

\subsection*{Format Masukan}
\addcontentsline{toc}{subsection}{Format Masukan} % for pdf indexing

Baris pertama berisi sebuah bilangan bulat T yang menyatakan banyaknya kasus uji. Baris-baris berikutnya berisi T kasus uji, yang masing-masing diberikan dalam format berikut ini:

\begin{lcverbatim}
N
P[1] L[1] R[1] K[1]
P[2] L[2] R[2] K[2]
.
.
P[N] L[N] R[N] K[N]
\end{lcverbatim}

\subsection*{Format Keluaran}
\addcontentsline{toc}{subsection}{Format Keluaran} % for pdf indexing

Untuk setiap kasus uji, keluarkan:

\begin{lcverbatim}
G[1]
G[2]
.
.
G[N]
\end{lcverbatim}

\vspace{.4cm}

\begin{minipage}[t]{0.5\textwidth}
\subsection*{Contoh Masukan}
\addcontentsline{toc}{subsection}{Contoh Masukan} % for pdf indexing

\begin{lcverbatim}
2
4
1 1 1 1
1 3 3 2
1 2 2 1
2 1 1 1
4
1 1 5 2
1 11 15 1
1 6 10 2
2 1 15 1
\end{lcverbatim}
\end{minipage}
\begin{minipage}[t]{0.5\textwidth}
\subsection*{Contoh Keluaran}
\addcontentsline{toc}{subsection}{Contoh Keluaran} % for pdf indexing

\begin{lcverbatim}
2
2
4
1
6
6
16
11
\end{lcverbatim}
\end{minipage}

\textit{Perhatikan bahwa contoh kedua tidak termasuk dalam contoh masukan dan contoh keluaran dari soal versi mudah.}

\subsection*{Penjelasan}
\addcontentsline{toc}{subsection}{Penjelasan} % for pdf indexing

Pada contoh pertama, mula-mula tidak terdapat bilangan pada papan tulis. Terdapat 4 aksi yang dilakukan oleh Pak Grandi:

\begin{itemize}
	\item Menuliskan 1 buah bilangan 1: di papan tulis terdapat [1], sehingga G[1] = 2.
	\item Menuliskan 2 buah bilangan 3: di papan tulis terdapat [1 3 3], sehingga G[2] = 2.
	\item Menuliskan 1 buah bilangan 2: di papan tulis terdapat [1 3 3 2], sehingga G[3] = 4.
	\item Menghapus 1 buah bilangan 1: di papan tulis terdapat [3 3 2], sehingga G[4] = 1.
\end{itemize}

\subsection*{Batasan}
\addcontentsline{toc}{subsection}{Batasan} % for pdf indexing

\begin{minipage}[t]{0.47\textwidth}

Batasan yang berlaku untuk versi mudah dan versi sulit:

\begin{itemize}
	\item 1 $ \leq $ T $ \leq $ 10
	\item 1 $ \leq $ P[i] $ \leq $ 2
	\item 1 $ \le $ L[i], R[i] $ \le $ 1.000.000.000
	\item 1 $ \le $ K[i] $ \le $ 1.000.000.000
	\item 1 $ \le $ N $ \le $ 50.000
\end{itemize}
\end{minipage}
\begin{minipage}[t]{0.06\textwidth}
	\hfill
\end{minipage}
\begin{minipage}[t]{0.47\textwidth}
Batasan khusus versi mudah:
\begin{itemize}
	\item L[i] = R[i]
\end{itemize}

\vspace{.2cm}

Batasan khusus versi sulit:
\begin{itemize}
	\item L[i] $ \le $ R[i]
\end{itemize}
\end{minipage}

\end{document}
