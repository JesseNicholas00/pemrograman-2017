% !TeX root = ../main_editorial.tex
\documentclass[../main_editorial.tex]{subfiles} % Inherits definitions from parent .tex file.

% Per-problem variable definitions
\newcommand{\problemName}{Operasi Bitwise}
\newcommand{\problemWriter}{Alham Fikri Aji}
\newcommand{\problemEditorialWriter}{Alham Fikri Aji}
\newcommand{\problemTags}{\textit{ad-hoc}, \textit{constructive}}

\begin{document}

\begin{center}
    \section*{\problemName}
    \addcontentsline{toc}{section}{\problemName} % for pdf indexing
    
    \begin{tabular}{rl}
		Penulis soal : & \problemWriter \\
	    Penulis editorial : & \problemEditorialWriter \\
  		Tema : & \problemTags
    \end{tabular}
\end{center}

\subsection*{Catatan/Komentar}
\addcontentsline{toc}{subsection}{Catatan/Komentar} % for pdf indexing

Untuk pembahasan soal ini, kita definisikan solusi sebagai $S_1, S_2, ..., S_N$.

\subsection*{Versi Mudah}
\addcontentsline{toc}{subsection}{Versi Mudah} % for pdf indexing
Batasan: $N = 2; 0 \le A, B, C \le 2^{20}$

Ada beberapa solusi yang dapat digunakan untuk menyelesaikan versi mudah. Salah satunya adalah dengan melakukan brute-force. Perhatikan bahwa N selalu 2. Lakukan perulangan untuk mencoba seluruh nilai $S_1$ dari 0 sampai $B$. Perulangan cukup dilakukan hingga $B$, karena nilai hasil operasi bitwise OR dua bilangan selalu lebih besar atau sama dengan hasil operasi AND ataupun XOR-nya. Untuk setiap nilai $S_1$, kita akan mencoba menghitung $S_2$. Perhatikan bahwa $S_1$ XOR $S_2 = C$, dan berdasarkan sifat XOR maka kita dapat menghitung $S_2 = C$ XOR $S_1$. Setelah didapatkan $S_1$ dan $S_2$, kita cukup melakukan validasi terhadap $A$ dan $B$. Kompleksitas solusi ini adalah $O(B)$.

\subsection*{Versi Sulit}
\addcontentsline{toc}{subsection}{Versi Sulit} % for pdf indexing
Batasan: $1 \le N \le 50.000; 0 \le A, B, C \le 2^{31}$

Mari kita bagi persoalan ini menjadi 3 kasus.
\subsubsection*{Kasus N = 1}
\addcontentsline{toc}{subsection}{Kasus N = 1} % for pdf indexing

$S_1 = A$ jika dan hanya jika $A = B = C$. Selain itu tidak ada solusi.

\subsubsection*{Kasus N = 2}
\addcontentsline{toc}{subsection}{Kasus N = 2} % for pdf indexing

Bisa diselesaikan sesuai pembahasan di versi mudah.

\subsubsection*{Kasus N $ > $ 2}
\addcontentsline{toc}{subsection}{Kasus N > 2} % for pdf indexing

Perhatikan bahwa setiap bit dapat kita selesaikan secara independen. Artinya, untuk setiap $0 \le i \le 31$, kita dapat mengerjakan soal ini dengan nilai baru $A_i, B_i, C_i$ yang mana $A_i$ bernilai 1 jika dan hanya jika $(A \mathrm{AND} 2^i) > 0 $, $B_i$ bernilai 1 jika dan hanya jika $(B \mathrm{AND} 2^i) > 0 $, dan $C_i$ bernilai 1 jika dan hanya jika $(C \mathrm{AND} 2^i) > 0 $. Untuk setiap $i$, kita akan mencari solusi $s[i]_1, s[i]_2, ..., s[i]_N$. Pada akhirnya, kita dapat menggabungkan solusi-solusi sebagai $$S_j \sum_{i=0}^{31} 2^i \times s[i]_j$$

Untuk setiap nilai $i$, kita tahu hanya ada 8 kemungkinan $A_i, B_i, C_i$ berbeda. Selain itu, kita bisa simpulkan bahwa nilai $A_i$ akan 1 jika seluruh $s[i]_j = 1$, nilai $B[i]_1$ akan 1 jika paling tidak ada 1 dari nilai $j$ yang memenuhi $s[i]_j = 1$, dan nilai $C_i$ akan 1 jika $j$ yang memenuhi $s[i]_j = 1$ berjumlah ganjil. Maka, kita dapat mencoba seluruh kemungkinan dan mendapatkan solusi sebagai berikut:

\begin{table}[H]
\centering
\begin{tabular}{|c|c|c|c|c|c|}
\hline
$ A_i $ & $ B_i $ & $ C_i $ & $ s[i]_1 $ & $ s[i]_2 $ & $ s[i]_3 \mathellipsis s[i]_N $       \\ \hline
0    & 0    & 0    & 0           & 0           & 0                                 \\ \hline
0    & 0    & 1    & \multicolumn{3}{c|}{tidak ada solusi}                          \\ \hline
0    & 1    & 0    & 1           & 1           & 0                                 \\ \hline
0    & 1    & 1    & 1           & 0           & 0                                 \\ \hline
1    & 0    & 0    & \multicolumn{3}{c|}{tidak ada solusi}                          \\ \hline
1    & 0    & 1    & \multicolumn{3}{c|}{tidak ada solusi}                          \\ \hline
1    & 1    & 0    & 1           & 1           & 1, hanya ada solusi jika N genap  \\ \hline
1    & 1    & 1    & 1           & 1           & 1, hanya ada solusi jika N ganjil \\ \hline
\end{tabular}
\end{table}

\end{document}