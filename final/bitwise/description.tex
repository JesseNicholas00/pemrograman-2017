% !TeX root = ../main_problemset.tex
\documentclass[../main_problemset.tex]{subfiles} % Inherits definitions from parent .tex file.

% Per-problem variable definitions
\newcommand{\problemName}{B. Operasi Bitwise}
\newcommand{\problemTL}{2 s}
\newcommand{\problemML}{256 MB}

\begin{document}

\begin{center}
    \section*{\problemName}
    \addcontentsline{toc}{section}{\problemName} % for pdf indexing
    
    \begin{tabular}{rl}
    Batasan waktu : & \problemTL \\
    Batasan memori : & \problemML
    \end{tabular}
\end{center}

\subsection*{Deskripsi}
\addcontentsline{toc}{subsection}{Deskripsi} % for pdf indexing

Carilah N bilangan bulat S[1], S[2], ..., S[N] yang memenuhi seluruh syarat berikut ini:

\begin{itemize}
	\item 0 $ \le $ S[i] $ < 2^{31} $
	\item S[1] AND S[2] AND ... AND S[N] = A
	\item S[1] OR S[2] OR ... OR S[N] = B
	\item S[1] XOR S[2] XOR ... XOR S[N] = C
\end{itemize}

dengan AND, OR, XOR merupakan operasi \textit{bitwise} \texttt{\&}, \texttt{|}, dan \texttt{\textasciicircum} secara berturut-turut pada C, C++, maupun Java.

\subsection*{Format Masukan}
\addcontentsline{toc}{subsection}{Format Masukan} % for pdf indexing

Baris pertama berisi sebuah bilangan bulat T yang menyatakan banyaknya kasus uji. Baris-baris berikutnya berisi T kasus uji, yang masing-masing diberikan dalam format berikut ini:

\begin{lcverbatim}
N A B C
\end{lcverbatim}

\subsection*{Format Keluaran}
\addcontentsline{toc}{subsection}{Format Keluaran} % for pdf indexing

Untuk setiap kasus uji:

Jika tidak mungkin terdapat N bilangan bulat yang memenuhi seluruh syarat tersebut, keluarkan:

\begin{lcverbatim}
-1
\end{lcverbatim}

Jika mungkin, keluarkan:

\begin{lcverbatim}
S[1] S[2] .. S[N]
\end{lcverbatim}

Jika terdapat lebih dari satu solusi, keluarkan yang mana saja.

\vspace{.4cm}

\begin{minipage}[t]{0.5\textwidth}
\subsection*{Contoh Masukan}
\addcontentsline{toc}{subsection}{Contoh Masukan} % for pdf indexing

\begin{lcverbatim}
3
2 8 14 6
2 4 6 6
3 4 6 6
\end{lcverbatim}
\end{minipage}
\begin{minipage}[t]{0.5\textwidth}
\subsection*{(Salah Satu) Contoh Keluaran}
\addcontentsline{toc}{subsection}{Contoh Keluaran} % for pdf indexing

\begin{lcverbatim}
14 8
-1
4 4 6
\end{lcverbatim}
\end{minipage}

\textit{Perhatikan bahwa contoh ketiga tidak termasuk dalam contoh masukan dan contoh keluaran dari soal versi mudah.}



\subsection*{Batasan}
\addcontentsline{toc}{subsection}{Batasan} % for pdf indexing

\begin{minipage}[t]{0.47\textwidth}
	
Batasan yang berlaku untuk versi mudah dan versi sulit:

\begin{itemize}
	\item 1 $ \leq $ T $ \leq $ 10
\end{itemize}
\end{minipage}
\begin{minipage}[t]{0.06\textwidth}
	\hfill
\end{minipage}
\begin{minipage}[t]{0.47\textwidth}
Batasan khusus versi mudah:
\begin{itemize}
	\item N = 2
	\item 0 $ \le $ A, B, C $ < 2^{20} $
\end{itemize}

\vspace{.2cm}

Batasan khusus versi sulit:
\begin{itemize}
	\item 1 $ \le $ N $ \le $ 50.000
	\item 0 $ \le $ A, B, C $ < 2^{31} $
\end{itemize}
\end{minipage}

\end{document}
