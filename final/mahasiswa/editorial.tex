% !TeX root = ../main_editorial.tex
\documentclass[../main_editorial.tex]{subfiles} % Inherits definitions from parent .tex file.

% Per-problem variable definitions
%TODO: Change placeholders to correct info.
\newcommand{\problemName}{Organisasi Kemahasiswaan}
\newcommand{\problemWriter}{Alham Fikri Aji}
\newcommand{\problemEditorialWriter}{Alham Fikri Aji}
\newcommand{\problemTags}{math, greedy}

\begin{document}

\begin{center}
    \section*{\problemName}
    \addcontentsline{toc}{section}{\problemName} % for pdf indexing
    
    \begin{tabular}{rl}
    Penulis soal : & \problemWriter \\
    Penulis editorial : & \problemEditorialWriter \\
    Tema : & \problemTags
    \end{tabular}
\end{center}

\subsection*{Catatan/Komentar}
\addcontentsline{toc}{subsection}{Catatan/Komentar} % for pdf indexing

Batasan kedua versi:
\begin{itemize}
    \item $1 \le K \le N$
    \item $1 \le M[i] \le 100.000$
\end{itemize}

\subsection*{Versi Mudah}
\addcontentsline{toc}{subsection}{Versi Mudah} % for pdf indexing
Batasan: $1 \le N \le 2$

Karena hanya ada paling banyak 2 organisasi, maka kita bisa selesaikan persoalan ini dengan melihat setiap kasus.
Jika hanya ada tepat 1 organisasi, maka dipastikan mahasiswa minimum yang mungkin adalah $M[1]$. Jika ada 2 organisasi, maka kita harus melihat nilai $K$.
Jika $K = 1$, maka setiap mahasiswa hanya boleh mengikuti paling banyak 1 organisasi, sehingga mahasiswa minimum yang mungkin adalah $M[1] + M[2]$. Namun jika $K = 2$, maka mahasiswa minimum yang mungkin adalah $\max(M[1],M[2])$

\subsection*{Versi Sulit}
\addcontentsline{toc}{subsection}{Versi Sulit} % for pdf indexing

Batasan: $1 \le N \le 100.000$

Cukup mudah disimpulkan bahwa mahasiswa paling sedikit berjumlah $$\max_{1 \le i \le N} M[i]$$ atau sama dengan anggota dari organisasi dengan jumlah anggota terbanyak.

Jika asumsikan Fasilkom memiliki total $T$ mahasiswa, dan setiap mahasiswa boleh mengikuti paling banyak $K$ organisasi, dapat disimpulkan bahwa $(\sum_{i=1}^N M[i]) \le T \times K$ harus terpenuhi. Jika tidak, menurut teorema pigeon-hole principle paling tidak ada 1 mahasiswa yang mengikuti lebih dari $K$ organisasi. Dengan demikian, kita dapat menghitung mahasiswa Fasilkom dengan rumus $$T = \left \lceil{\frac{\sum_{i=1}^N M[i]}{K}}\right \rceil$$ dan jawaban akhir berupa $$\max \left ( \max_{1 \le i \le N} M[i],\left \lceil{\frac{\sum_{i=1}^N M[i]}{K}}\right \rceil \right )$$

\end{document}