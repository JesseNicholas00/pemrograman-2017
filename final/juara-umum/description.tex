% !TeX root = ../main_problemset.tex
\documentclass[../main_problemset.tex]{subfiles} % Inherits definitions from parent .tex file.

% Per-problem variable definitions
\newcommand{\problemName}{G. Juara Umum}
\newcommand{\problemTL}{2 s}
\newcommand{\problemML}{64 MB}

\begin{document}

\begin{center}
    \section*{\problemName}
    \addcontentsline{toc}{section}{\problemName} % for pdf indexing
    
    \begin{tabular}{rl}
    Batasan waktu : & \problemTL \\
    Batasan memori : & \problemML
    \end{tabular}
\end{center}

\subsection*{Deskripsi}
\addcontentsline{toc}{subsection}{Deskripsi} % for pdf indexing

Tahun ini, Gemastik terdiri atas N cabang lomba, yang dinomori dari 1 hingga N. Terdapat M universitas sebagai peserta Gemastik, yang dinomori dari 1 hingga M. Setiap universitas mengirimkan tepat satu tim untuk setiap cabang lomba.

Untuk setiap cabang lomba, akan diambil tiga tim berbeda sebagai pemenang, yang berturut-turut akan memperoleh medali emas, perak, dan perunggu. Setelah semua cabang lomba dipertandingkan, peringkat seluruh universitas akan diurutkan berdasarkan perolehan medali. Universitas A akan meraih peringkat yang lebih baik daripada universitas B, apabila:

\begin{itemize}
	\item perolehan emas A $ > $ perolehan emas B, atau
	\item perolehan emas A = perolehan emas B dan perolehan perak A $ > $ perolehan perak B, atau
	\item perolehan emas A = perolehan emas B, perolehan perak A = perolehan perak B, dan perolehan perunggu A $ > $ perolehan perunggu B.
\end{itemize}

Akhirnya, universitas yang memiliki peringkat terbaik akan mendapatkan gelar juara umum.

Perhatikan bahwa menurut sistem pemeringkatan di atas, mungkin saja terdapat lebih dari satu juara umum. Mungkinkah terdapat tepat K universitas sebagai juara umum?

\subsection*{Format Masukan}
\addcontentsline{toc}{subsection}{Format Masukan} % for pdf indexing

Baris pertama berisi sebuah bilangan bulat T yang menyatakan banyaknya kasus uji. Baris-baris berikutnya berisi T kasus uji, yang masing-masing diberikan dalam format berikut ini:

\begin{lcverbatim}
N M K
\end{lcverbatim}

\subsection*{Format Keluaran}
\addcontentsline{toc}{subsection}{Format Keluaran} % for pdf indexing

Untuk setiap kasus uji:

Jika tidak mungkin terdapat tepat K universitas sebagai juara umum, keluarkan:

\begin{lcverbatim}
mustahil
\end{lcverbatim}

Jika mungkin, keluarkan:

\begin{lcverbatim}
mungkin
G[1] S[1] B[1]
G[2] S[2] B[2]
.
.
G[N] S[N] B[N]
\end{lcverbatim}

\vspace{.4cm}

dengan G[i], S[i], B[i] masing-masing merupakan nomor universitas yang mendapatkan medali emas, perak, dan perunggu pada cabang lomba ke-i. Jika terdapat lebih dari satu kemungkinan, keluarkan yang mana saja.

\begin{minipage}[t]{0.5\textwidth}
\subsection*{Contoh Masukan}
\addcontentsline{toc}{subsection}{Contoh Masukan} % for pdf indexing

\begin{lcverbatim}
4
3 3 2
5 10 2
4 3 3
8 6 3
\end{lcverbatim}
\end{minipage}
\begin{minipage}[t]{0.5\textwidth}
\subsection*{(Salah Satu) Contoh Keluaran}
\addcontentsline{toc}{subsection}{Contoh Keluaran} % for pdf indexing

\begin{lcverbatim}
mustahil
mungkin
1 9 10
1 9 10
2 6 7
2 6 7
5 9 10
mustahil
mungkin
1 2 6
1 2 6
2 3 6
2 3 6
3 1 6
3 1 6
4 5 6
5 4 6
\end{lcverbatim}
\end{minipage}

\textit{Perhatikan bahwa contoh ketiga dan keempat tidak termasuk dalam contoh masukan dan contoh keluaran dari soal versi mudah.}

\subsection*{Penjelasan}
\addcontentsline{toc}{subsection}{Penjelasan} % for pdf indexing

Untuk contoh kedua, universitas 1 dan universitas 2 menjadi juara umum bersama, karena masing-masing memperoleh 2 medali emas.

Untuk contoh keempat, universitas 1, 2, dan 3 menjadi juara umum bersama, karena masing-masing memperoleh 2 medali emas dan 2 medali perak.

\subsection*{Batasan}
\addcontentsline{toc}{subsection}{Batasan} % for pdf indexing

\begin{minipage}[t]{0.47\textwidth}

Batasan yang berlaku untuk versi mudah dan versi sulit:

\begin{itemize}
	\item 1 $ \leq $ T $ \leq $ 10
	\item 1 $ \leq $ N $ \leq $ 50.000
	\item 3 $ \le $ M $ \le $ 50.000
	\item K $ \le $ N
\end{itemize}
\end{minipage}
\begin{minipage}[t]{0.06\textwidth}
	\hfill
\end{minipage}
\begin{minipage}[t]{0.47\textwidth}
Batasan khusus versi mudah:
\begin{itemize}
	\item K = 2
\end{itemize}

\vspace{.2cm}

Batasan khusus versi sulit:
\begin{itemize}
	\item 1 $ \le $ K $ \le $ M
\end{itemize}
\end{minipage}

\end{document}
