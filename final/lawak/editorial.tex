% !TeX root = ../main_editorial.tex
\documentclass[../main_editorial.tex]{subfiles} % Inherits definitions from parent .tex file.

% Per-problem variable definitions
\newcommand{\problemName}{Berbagi Lawakan}
\newcommand{\problemWriter}{William Gozali}
\newcommand{\problemEditorialWriter}{William Gozali}
\newcommand{\problemTags}{\textit{graph}, \textit{memoization}}

\newcommand{\tnode}{\textit{node}\xspace}
\newcommand{\tNode}{\textit{Node}\xspace}
\newcommand{\tcycle}{\textit{cycle}\xspace}
\newcommand{\tCycle}{\textit{Cycle}\xspace}
\newcommand{\bigO}[1]{\mathcal{O}(#1)} % stolen from Soko

\begin{document}

\begin{center}
    \section*{\problemName}
    \addcontentsline{toc}{section}{\problemName} % for pdf indexing
    
    \begin{tabular}{rl}
    Penulis soal : & \problemWriter \\
    Penulis editorial : & \problemEditorialWriter \\
    Tema : & \problemTags
    \end{tabular}
\end{center}

\subsection*{Catatan/Komentar}
\addcontentsline{toc}{subsection}{Catatan/Komentar} % for pdf indexing

Batasan kedua versi: $1 \le N, M, Q \le 50000$

\subsection*{Versi Mudah}
\addcontentsline{toc}{subsection}{Versi Mudah} % for pdf indexing
Batasan: $\mathbf{A[i] = 0; B[i] = 100}$

Tugas kita adalah menentukan apakah suatu \tnode akan menuju ke suatu \tcycle.

Cara sederhananya adalah melakukan \textit{Depth First Search} (DFS) mulai dari \tnode yang ditanyakan, yaitu \tnode $x$. Misalkan sejauh ini DFS telah membentuk \textit{path} dari $x$ sampai \tnode $p$. Untuk seluruh \tnode $q$ yang dapat dikunjungi langsung dari $p$, periksa apakah $q$ merupakan anggota dari \textit{path} yang dibentuk dari $x$ sampai $p$. Apabila ya, berarti lawakan dimulai dari \tnode $x$ menjadi abadi, dan tidak untuk kasus sebaliknya.

Hasil penelusuran dari DFS dapat digunakan untuk menjawab pertanyaan berikutnya. Untuk setiap \tnode, catat informasinya sebagai salah satu dari nilai berikut:
\begin{itemize}
  \item 0, bila \tnode ini belum diketahui apakah ke depannya akan berakhir pada \tcycle.
  \item 1, bila \tnode ini sudah diketahui akan berakhir pada \tcycle.
  \item 2, bila \tnode ini sudah diketahui tidak akan berakhir pada \tcycle.
\end{itemize}

Awalnya seluruh \tnode meiliki informasi bernilai 0.

Saat melakukan DFS untuk menjawab pertanyaan berikutnya, jika ditemui \tnode yang memiliki informasi bernilai 1 atau 2, maka DFS dapat dihentikan dan jawaban langsung ditemukan.

Kompleksitas waktu solusi ini adalah $\bigO{M + N + Q}$.

\subsection*{Versi Sulit}
\addcontentsline{toc}{subsection}{Versi Sulit} % for pdf indexing

Batasan: $\mathbf{0 \le A[i] \le B[i] \le 100}$

Versi ini dapat diselesaikan dengan solusi pada versi mudah, tetapi dengan mempertimbangkan bahwa kali ini terdapat paling banyak $100$ \textit{graph} yang berbeda.

Anda dapat menggunakan \textit{array} sebesar $100 \times 50000$ untuk menampung informasi dari tiap \tnode, atau memproses seluruh pertanyaan dari yang nilai humornya paling rendah ke paling besar. Kompleksitas waktu solusi ini adalah $\bigO{Q + 100(N + M)}$, atau tetap dapat dituliskan dengan $\bigO{M + N + Q}$.

\end{document}